\section{Evaluation}
\label{sec:Evaluation}
\subsection{Uncertainty base method}
In  Figure \ref{fig:uncertainty1} we can see that in the grand scale, in the uncertainty base methods behave similarly with random method, however when we looks closer at the half end of the left figure, most of the uncertainty base method gives 1\% improvement in accuracy compare with the random method.   

In this experiment, we have tried to advoid the use of core-set and adversarial approaches 
%When evaluating your results, avoid drawing grand conclusions, beyond that which your results can in fact support. Further, although you may have designed your experiments to answer certain questions, the results may raise other questions in the eyes of the reader. It is important that you study the graphs/tables to look for unusual features/entries and discuss these as well as discussing the main findings in the results. 
\subsection{Distance based method}:
In this project, we have also tried to apply the core set 
\section{Discussion}
\label{sec:Discussion}
\section{Uncertainty base active learning and Bald}
Most of the uncertainty base active learning suffers heavily when run batch incremental mode. It is not a surprise since for large batch size, the instances that give the same uncertainty output tends to belong to the same class or very similar to each other. Because of this overlapping, the uncertainty base active learner will not able to query instances in a diversify manner, which lead to poorer performance. 
\subsection{The effect of batch-size during incremental learning}
In  Figure \ref{fig:uncertainty1} we can see 


\section{Contributions}~\label{cont}
\label{sec:Contributions}
For the situation where we gradually increase the size of the train set. If 

In the discussion it is important to include a discussion of not just the merits of the work conducted but also the limitations. 
What are the main contributions made to the field and how significant are these contribution.  

\section{Future Work}
\label{sec:futureWork}

Consider where you would like to extend this work. These extensions might either be continuing the ongoing direction or taking a side direction that became obvious during the work. Further, possible solutions to limitations in the work conducted, highlighted in ~\ref{sec:Discussion} may be presented. 
